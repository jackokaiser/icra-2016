\documentclass{article}

%\documentstyle[12pt]{book}
\begin{document}

\begin{center}

\large {\bf Simultaneous State Initialization and Gyroscope Bias Calibration in Visual Inertial aided Navigation}\\

\vskip.2cm

Responses to Reviewer Comments\\

Jacques Kaiser, Agostino Martinelli, Flavio Fontana and Davide Scaramuzza
\end{center}

\vskip.5cm

 \small

\noindent We would like to thank the editor and the reviewers for
the time they spent reading and carefully reviewing the manuscript, and for
their constructive suggestions, which helped us to improve the paper.
Below, is our response to the reviewers' remarks
with the changes that we've made to the document.





\vskip.5cm




%%%%%%%%%%%%%%%%%%%%%%%%%%%%%%%%%%%%%%%%%%%%%%%%%%%%
%%%%%%%%%%%%%%%%%%%%%%%%%%%%%%%%%%%%%%%%%%%%%%%%%%%%
%%%%%%%%%%%%%%%%%%%%%%%%%%%%%%%%%%%%%%%%%%%%%%%%%%%%
\noindent {\bf Reviewer \#1:} \\ {\it The authors study empirically the impact of noise on a recently derived
closed-form solution [1] that determines the angular and linear
velocity of a moving platform by fusing visual and inertial
measurements. The main contributions, as claimed by the authors, are
twofold: (a) gyro bias is (empirically) shown to have significant
impact on performance; (b) a new method is introduced for online
estimation of the gyro bias.

Overall, the paper is well-written, tries to be self-contained, and is
easy to follow. However, in my opinion, it is not sufficiently
innovative and lacks thorough analysis to meet the high-standards of
RA-L. On the other hand, the topic is important and the presented
results and proposed improvement are impressive, and thus are of
interest to the robotics community. Therefore, the paper is suitable
for publication in ICRA.

Detailed comments follow below.

The authors generally do a good job addressing related work in
visual-inertial aided navigation and visual-inertial structure from
motion, mentioning both filtering and optimization based approaches. I
would suggest to add also the following related publications:

- Huang, Guoquan, Michael Kaess, and John J. Leonard. "Towards
consistent visual-inertial navigation." In Robotics and Automation
(ICRA), 2014 IEEE International Conference on, pp. 4926-4933. IEEE,
2014.

- Indelman, Vadim, Stephen Williams, Michael Kaess, and Frank Dellaert.
"Information fusion in navigation systems via factor graph based
incremental smoothing." Robotics and Autonomous Systems 61, no. 8
(2013): 721-738.

- Hesch, Joel, Dimitrios G. Kottas, Sean L. Bowman, and Stergios
Roumeliotis. "Consistency analysis and improvement of vision-aided
inertial navigation." Robotics, IEEE Transactions on 30, no. 1 (2014):
158-176.

}

\vskip.3cm
\noindent {\bf Answer \#1.1:} These references have been included.


\vskip.3cm


\noindent {\it
The paper builds upon [1] and shows empirically, using both simulated
and real-world data, that gyroscope bias has significant impact of
performance of the closed-form solution of [1]. My main concern,
however, is that in its current version the study is purely empirical
and does not attempt to investigate sensitivity of the closed-form
solution using more analytical tools. This would shed further light, I
believe, to the observation made by the authors, and allow addressing
my next comment.

}

\vskip.3cm
\noindent {\bf Answer \#1.2:}
We agree that most of the results have been obtained empirically.
On the other hand, obtaining analytic results that describe the sensitivity of the performance on the gyro Bias is ambitious.
Indeed, the gyro bias in (1) cannot be separated from the true rotation and this because rotations are in general non-commutative.
Additionally, even if true rotations do not occur, the change of the left-hand side in (1) due to the bias, is non-trivial for a generic acceleration.
Similarly, providing analytic properties of the cost function is also ambitious since it depends on the entire trajectory. See also the next answer  \#1.4
\vskip.3cm


\noindent {\it
One source of confusion is that the impact of gyroscope bias on
performance was already considered in Monte-Carlo study in [1] (section
5.2). This fact should be mentioned in the present paper and the
difference between the two studies should be discussed. In particular,
it seems that the conclusion regarding the significant impact of
gyroscope bias on performance should have been reached in both cases;
however that appears to be not the case. A discussion of these aspects
would be beneficial, in my opinion.
}

\vskip.3cm
\noindent {\bf Answer \#1.3:}  We explicitly mentioned the differences with [1] at the beginning of section IV and in the last paragraph of  IV.D.
Specifically, unlike [1], we consider plausible drone motion and a larger gyro bias.
The gyro bias considered in [1] was too small to detect its impact.


\vskip.3cm


\noindent {\it
Another point that deserves clarification is the way the cost function
in eq. 4 is minimized. Since that optimization process requires
initialization, can it be stuck in local minima? An intuitive
explanation why the symmetry in the cost function is a bad thing would
be also beneficial (Section V-B).
}

\vskip.3cm
\noindent {\bf Answer \#1.4:} It is true. This can occur in principle.
Extensive simulations allowed us to conjecture that the function is locally convex (i.e., in the region around the true bias).
This has now been mentioned in section V.A. If the time of integration is small, and the rotations only occur around a single axis, this local convexity is lost.
In particular, the cost function exhibits a symmetry: it is invariant with respect to the component of the bias collinear with the gravity.
An explanation why the symmetry can yield the optimization process to converge towards local minima in the cost function has been added in the last paragraph of V.A.
\vskip.3cm


\noindent {\it
Minor comment - please indicate where [19] was published.
}

\vskip.3cm
\noindent {\bf Answer \#1.5:} Done

\newpage
%%%%%%%%%%%%%%%%%%%%%%%%%%%%%%%%%%%%%%%%%%%%%%%%%%%%
%%%%%%%%%%%%%%%%%%%%%%%%%%%%%%%%%%%%%%%%%%%%%%%%%%%%
%%%%%%%%%%%%%%%%%%%%%%%%%%%%%%%%%%%%%%%%%%%%%%%%%%%%
\noindent {\bf Reviewer \#2:} \\ {\it This paper presents an algorithm which obtains closed-form estimates
for the initial values for the velocity and gravity vectors and sensor
biases (along with feature depths) for the problem of visual-inertial
localization. The algorithm is largely a modification of an existing
method, in which case the original closed-form solution was detailed.

The paper begins by briefly explaining the method in Ref. 1. Of
note is the specific emphasis on Eq. 10 in the original paper (Eq. 3 in
the current paper) where the gravity vector magnitude is constrained.

Sec. IV details limitations which the authors point out exist in the
original method. To show this, a combination of real and synthetic data
is used, although it is not clear why this is the case. It is mentioned
that the justification for using synthetic camera data is the
availability of ground-truth landmark depth values, for comparison.
However, it seems that by that logic it would also make sense to use
simulated pose and measurement data as well. This renders Figure 3
rather confusing. Presumably the ìrelative errorî axis is compared with
ground-truth data. However since the paper implies that real data was
used for position and inertial measurements, what was the source of the
ground truth data for these terms?

}

\vskip.3cm
\noindent {\bf Answer \#2.1:} Section IV has totally been re-written. Now all the measurements are synthetic, as suggested.
Previously, we were using real inertial data to ensure that the motion was plausible for a drone, with realistic sensor error.
The ground truth position was obtained from an optitrack motion capture system, following the same setup as for the validation.

\vskip.3cm
\noindent {\it Furthermore, it is unclear why the comparison in Fig. 3 is made. It
seems as though the figure serves to show that the gravity refinement
outlined in the original algorithm does not significantly improve the
result of the experiment. Is this used as a basis to remove this
constraint in the following sections? If so, please explicitly state
this. Itís also recommended to outline or discuss any cases where the
authors feel that constraining the magnitude of the gravity vector
could be beneficial.

}

\vskip.3cm
\noindent {\bf Answer \#2.2:} Now we clearly state at the end of section IV-B that we remove the gravity constraint after this section. Since the gravity is well-estimated all the time, we do not feel that constraining its magnitude could be beneficial.


\vskip.3cm
\noindent {\it
Sec. IV.C states that accelerometer bias (within the attempted
ranges) does not seriously undermine the original algorithm. The
authors also state that the variant of the original algorithm which
estimates the biases is not robust in the experiment due to
insignificant effect. It is recommended to expand on this as it is
unclear. Perhaps a comparison between the two variants? Does the
situation change if the biases are large?
}

\vskip.3cm
\noindent {\bf Answer \#2.3:} Actually, the bias on the accelerometer cannot be estimated for a motion where important rotations, at least around two independent axes, do not occur. Indeed, the accelerometer bias is non-observable if these rotations do not occur (see property 12 and/or table 2 in [1]). This has been mentioned in a footnote of section IV.C. Additionally, in the first paragraph of section IV.C, we provide an explanation for the negligible effect of the accelerometer bias.


\vskip.3cm
\noindent {\it
Finally it is shown that the gyroscope bias has a significant impact on
the solution. This reviewer feels that this may have more significance
if all the data were generated via simulation. The use of mixed data
(with inherent biases and errors) potentially undermines the precise
significance of a particular source of added error.
}

\vskip.3cm
\noindent {\bf Answer \#2.4:} We fully agree with this and we followed this suggestion: Section IV has been re-written using only simulations and synthetic measurements.
See answer \#2.1 for more details.


\vskip.3cm
\noindent {\it
Sec. V outlines the modifications made to the original algorithm to
estimate the gyro bias. The specific formulation is unclear to this
reviewer. Eq. 4 changes the original closed-form solution to an
optimization. It is not clear what form the original matrices take when
ìcomputed with respect to Bî. The paper hints that Eq. 4 is solved via
an optimization. Is the solution still closed form? Or do gradients
need to be computed to optimize the non-linear constraints? Please
consider adding detail to this section, as in this reviewer's opinion
it constitutes the majority of the theoretical contribution of the
work.

If the closed-form solution has indeed been turned into an iterative
optimization, what are the implications for performance? Please
consider adding performance metrics to the paper, as initialization is
a time-sensitive stage of visual-inertial estimation.

}

\vskip.3cm
\noindent {\bf Answer \#2.5:} The way we compute the matrices with respect to the gyro bias in Eq. (4) is now stated clearly.
The optimization process is no more a closed-form, see answer \#1.4.
As a performance metrics, we added the average number of time that the cost function has been evaluated in section V.a.



\vskip.3cm
\noindent {\it
Please consider adding a comparison metric to Figure 6. It is not clear
if the values that the solution converged to are correct. Were these
bias values the ones that were used to corrupt the measurements? What
about the pre-existing bias in those measurements since they were from
real sensor data?

}

\vskip.3cm
\noindent {\bf Answer \#2.6:} Now that we perform our evaluations on purely synthetic measurements,
we can precisely provide the ground truth bias. Moreover, the relative error between
the real bias and the bias estimate has been added.

\vskip.3cm
\noindent {\it
Sec. V.C outlines a certain symmetry in the cost function. This problem
is solved by adding a regularizer which penalizes large bias estimates.
To this reviewer,
this does not seem like a principled solution. It penalizes all
gyro bias axes equally it is possible to know which axis is continually
collinear with gravity, and regularize only that axis. This regularizer
could also potentially
result in erroneous estimates if sufficient excitation did exist during
initialization and the correct bias values were in fact very large.

If the regularizer is used to include prior information,
it seems that the value of lambda should be set given the covariance of
these prior estimates, rather than empirically.
}

\vskip.3cm
\noindent {\bf Answer \#2.7:} We really thank the reviewer for this remark.
The cost function has been modified accordingly: now only the component of the gyro bias colinear with the gravity is regularized (see equation (5)).
In the last paragraph of section V.C we also added that the value of the regularizer should be picked with respect to the range of change of the gyroscope bias.


\vskip.3cm
\noindent {\it
Section VI describes experiments using real data, where it is unclear
(in Fig. 11) whether there is any significant advantage in using the
modified algorithm. As the authors themselves state, this may be
because the estimated bias values are very small. The merits of the
proposed solution as opposed to the original algorithm would be more
convincing if an experiment was presented where the results were
significantly improved.
}

\vskip.3cm
\noindent {\bf Answer \#2.8:} We added two new figures where we corrupted the true measurements with two artificial biases. In fig 9b the artificial bas is 0.05 rad/s in fig 9c is 0.1 rad/sec. Now, the optimized solution significantly outperforms the original solution. This because the new solution estimated very accurately the bias and hence the performance is unaffected by this artificial bias.

\newpage

%%%%%%%%%%%%%%%%%%%%%%%%%%%%%%%%%%%%%%%%%%%%%%%%%%%%
%%%%%%%%%%%%%%%%%%%%%%%%%%%%%%%%%%%%%%%%%%%%%%%%%%%%
%%%%%%%%%%%%%%%%%%%%%%%%%%%%%%%%%%%%%%%%%%%%%%%%%%%%
\vskip.5cm
\noindent {\bf Editor: } \\ {\it In this paper, the authors consider the problem of extending the
algorithm in [1] for initialisation to consider real data and, in
particular, the impact of gyro biases. To address the issue with the
gyro biases, a regularisation scheme is introduced. Both reviewers feel
that the paper is well-written and interesting. However, they have the
following concerns:

1. Both reviewers express concerns about the lack of innovation in the
paper. R1 notes that the analysis is almost entirely empirical and
lacks any theoretical analysis. R2 argues that the work builds upon [1]
by adding a nonlinear optimization scheme and regulariser to compute
the gyro biases. However, there are many questions about the validity
and performance of this approach.
\vskip.3cm
\noindent {\bf Answer \#E.1:} See answers  \#1.2,  \#1.4, and  \#2.7


\vskip.3cm


\noindent {\it 2. R2 expresses the additional concern that the results mix both real
data and simulations because this can mix in unmodelled effects. This
raises the question of why the analysis in Section IV did not simply
use the output from a high-quality simulator.

}

\vskip.3cm
\noindent {\bf Answer \#E.2:} See answers \#2.1, \#2.4


\vskip.3cm


\noindent {\it 3. R1 expressed the concern that the paper did not fully describe the
study in section 5.2 of [1]. Checking this, the reviewer is correct in
that the study does include time varying biases on both the
accelerometers and the gyros jointly - results for each independently
of one another do not appear to be presented. This clarification is
unclear in the paper.
}

\vskip.3cm
\noindent {\bf Answer \#E.3:} See answer \#1.3

\vskip.3cm


\noindent {\it
4. R2 expressed concerns about the discussion in Section IV.C and
biases. The discussion in the section is unclear about the similarities
and differences between the algorithms and [1]. To me, it raises the
question of whether the accelerometer readings actually play a
significant role with the algorithms.

As regards the relative contribution of the biases, I think it's
reasonably clear why the example, in Fig. 4 and Fig. 5, would produce
very different-looking results. 0.2 rad/s is 11 deg/s drift, which is
clearly very large. $0.2m/s^2$ error leads, in my calculations to a worst
case estimated error in orientation of about 0.8 degrees
(atan2(0.2/sqrt(2),9.81-0.2/sqrt(2))*180/pi).
}

\vskip.3cm
\noindent {\bf Answer \#E.4:}  See answer \#2.3




%\vskip.3cm
%
%
%\noindent {\it
%On one hand, the main issues are raised in point 1 - the analysis
%needs to be more thorough and complete - but they are not expected to
%be fully solved in the revised version. On the other hand, the other
%more minor issues have to be correctly addressed.
%
%}
%
%\vskip.3cm
%\noindent {\bf Answer \#E.6:} ...




\end{document}
